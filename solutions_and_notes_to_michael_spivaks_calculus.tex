\documentclass[12pt]{article}
\usepackage{amsfonts}
\usepackage{amsmath}
\renewcommand\labelenumi{(\roman{enumi})}
\renewcommand\theenumi\labelenumi
\begin{document}
\title{Solutions and Notes to Michael Spivak's \emph{Calculus}}
\author{}
\date{}
\maketitle
\cleardoublepage

\section{Basic Properties of Numbers}
\subsection{}
Prove the following.
\subsubsection{}
If $ax=a$ for some number $ a\neq 0$, then $x=1$.
\begin{equation*}
\begin{split}
ax&=a\\ 
x&=\frac{a}{a}\\
x&=1
\end{split}
\end{equation*}
\subsubsection{}
$x^2-y^2=(x-y)(x+y)$
\begin{equation*}
\begin{split}
x^2 - y^2 &= x^2 + xy - xy - y^2\\ 
x^2 - y^2 &= x^2 - y^2
\end{split}
\end{equation*}

\subsection{}
What is wrong with the following "proof"? Let $x=y$.
\begin{equation*}
\begin{split}
(x+y)(x-y) &= y(x-y)\\ 
x + y &= y
\end{split}
\end{equation*}
We divide by $(x-y)$, which given $x=y$ equals $0$. We cannot divide by zero.

\subsection{}
Prove the following.

\subsubsection{}
$\frac{a}{b}=\frac{ac}{bc}$, if $b,c\neq 0$.
\begin{equation*}
\begin{split}
ab^{-1} &= acb^{-1}c^{-1}\\ 
ab^{-1} &= ab^{-1}
\end{split}
\end{equation*}

\subsubsection{}
$\frac{a}{b}+ \frac{c}{d}=\frac{ad+bc}{bd}$, if $b,d\neq 0$.

\begin{equation*}
\begin{split}
\frac{a}{b}+ \frac{c}{d} &= \frac{ad+bc}{bd}\\
ad+cb&=ad+cb
\end{split}
\end{equation*}

\subsubsection{}
$(ab)^{-1}= a^{-1}b^{-1}$, if $a,b \neq 0$.

\begin{equation*}
\begin{split}
(ab)^{-1} &= a^{-1}b^{-1}\\
\frac{1}{ab} &= \frac{1}{a}\frac{1}{b}\\
\frac{1}{ab} &= \frac{1}{ab}
\end{split}
\end{equation*}

\subsubsection{}

\subsubsection{}

\begin{equation*}
\begin{split}
\frac{\frac{a}{b}}{\frac{c}{d}} &= \frac{ad}{bc} \text{, if } b,c,d \neq 0\\
\frac{a}{b}\cdot (\frac{c}{d})^{-1} &=  \frac{ad}{bc}\\
ab^{-1}\cdot c^{-1} \cdot (d^{-1})^{-1} &=  \frac{ad}{bc}\\
ab^{-1}\cdot c^{-1} \cdot d &=  \frac{ad}{bc}\\
\frac{ad}{bc} &=  \frac{ad}{bc}
\end{split}
\end{equation*}

\subsubsection{}
If $b,d \neq 0$, then $\frac{a}{b} = \frac{c}{d}$ if and only if $ad=bc$. Also determine when $\frac{a}{b}=\frac{b}{a}$.\\\\
To prove a theorem of the form \emph{A if and only if B}, you first prove \emph{if A then B}, then you prove \emph{if B then A}, and that's enough to complete the proof.

\emph{if A then B}:

\begin{equation*}
\begin{split}
\frac{a}{b} &= \frac{c}{d}\\
ab^{-1} &= cd^{-1}\\
ab^{-1}bd &= cd^{-1}bd\\
ad &= bc
\end{split}
\end{equation*}

\emph{If B then A} is given by the inverse direction of the above.

\subsection{}
Find all numbers $x$ for which:

\subsubsection{}

\begin{equation*}
\begin{split}
4 - x &< 3 - 2x\\
x &<-1\\
x &\in \{(-1, \infty)\}
\end{split}
\end{equation*}

\subsubsection{}

\begin{equation*}
\begin{split}
5-x^2&<8\\
-3 &< x^2
\end{split}
\end{equation*}

Every real number squared is positive, hence $\in \{\mathbb{R}\}$

\subsubsection{}
\begin{equation*}
\begin{split}
5-x^2&<-2\\
-x^2 &< -7\\
x^2 &> 7\\
x &> \sqrt{7}
\end{split}
\end{equation*}

$x \in \{x> \sqrt{7}\}$

\subsubsection{}
\begin{equation*}
\begin{split}
(x-1)(x-3)>0\\
\text{then } x \neq 1 \text{ and } x \neq 3
\end{split}
\end{equation*}

$x \in \{x>3, x<1\}$

\subsubsection{}
$x^2-2x+2$ draws a parabola which is greater zero for all $x\in \{\mathbb{R}\}$.

\end{document}